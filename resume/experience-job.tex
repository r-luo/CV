%-------------------------------------------------------------------------------
%	SECTION TITLE
%-------------------------------------------------------------------------------
\cvsection{Work Experiences}


\begin{cventries}
\cventry
    {Senior Data Scientist} % Job title
    {StackAdapt} % Organization
    {Toronto, Ontario, Canada} % Location
    {Jan. 2021 - Present} % Date(s)
    {
      \begin{cvitems} % Description(s) of tasks/responsibilities
        \item {Analyzed performance lift provided by additional features on existing models. The models are trained with dataset size in the magnitude of terabytes with Scala and Spark on DataBricks. Utilized SHAP values to interpret the signals captured by the model}
        \item {Investigating the effects of weekdays on click through rate and conversion rate across advertisement campaigns}
        \item {Exploring methods to summarize contents on webpages using pretrained NLP models}
        \item {Introduced facebook Prophet model on bid volume prediction, which offered better performance over existing models}
      \end{cvitems}
    }

\end{cventries}


%-------------------------------------------------------------------------------
%	CONTENT
%-------------------------------------------------------------------------------
\begin{cventries}
\cventry
    {Data Scientist II | Sponsored Product Advertising} % Job title
    {Amazon} % Organization
    {Toronto, Ontario, Canada} % Location
    {Jul. 2020 - Nov. 2020} % Date(s)
    {
      \begin{cvitems} % Description(s) of tasks/responsibilities
        \item {Built a pipeline that streamlines data processing, model build and validation to accelerate the worldwide expansion of a project from the planned 4 months to 1 month. Produced validation analyses and presented to the product team}
        \item {Mentored a Business Intelligence Engineer to help his transition to Data Scientist}
      \end{cvitems}
    }

\end{cventries}


\begin{cventries}

%---------------------------------------------------------
  \cventry
    {Principal Data Scientist | Fraud Team} % Job title
    {Capital One Bank} % Organization
    {Toronto, Ontario, Canada} % Location
    {Jun. 2018 - Jun 2020} % Date(s)
    {
      \begin{cvitems} % Description(s) of tasks/responsibilities
        \item {Leading data science projects for application fraud.}\\
        \begin{cvitems}
          \item {Migrated current model from legacy platform to Python on AWS as part of our enterprise-wide cloud migration.}
          \item {Refitted the model with LightGBM. Independently finished model build from data collection to deployment, designed the downstream business policy and provided valuation for model impact on business budgeting. The refit reduced expected loss by 40\% and false positive rate by 20\%.}
          \item {Utilized hyperparameter tuning and feature selection techniques to improve model performance. Developed a Markov Chain Monte Carlo based hyperparameter search algorithm which avoids the overhead of popular Bayesian based hyperparameter optimization techniques for smaller sized problems.}
          \item {Explored semi-supervised learning methods to utilize unlabeled data. As a result, our models adapt better to recent trends in the population.}
          \item {Supervised a co-op student on the deployment of a real time model on our streaming platform.}
          \item {Supervised a second co-op student to modernize our fraud defence valuation pipeline. It reduced the time needed for developing new fraud rules from weeks to hours. Received great feedbacks from both coops.}
          \item {Prototyped a data pipeline as a python module to bring model development and deployment into one place.}\\
        \end{cvitems}
        \item {Started a journal club within the data scientist community, facilitated bi-weekly paper discussions to bring the team up-to-date with the academia, and to drive team engagement and innovation.}
        \item {Promoted to Principal Data Scientist in July 2019.}
      \end{cvitems}
    }

    \cventry
      {Senior Data Scientist | Customer Management Team} % Job title
      {} % Organization
      {} % Location
      {Sep. 2015 - May. 2018} % Date(s)
      {
        \begin{cvitems} % Description(s) of tasks/responsibilities
          \item {Built a core business model and its end-to-end scoring pipeline for predicting customer default.}\\
          \begin{cvitems}
            \item {The whole process involved sample selection, data pull, data cleaning and validation, feature engineering and selection, model build and validation, model deployment, documentation, and ongoing model monitoring.}
            \item {Used SQL, SAS and Tableau for data cleaning and validation; used GBM and Bayesian model selection techniques in R for feature selection; explored genetic algorithm along the way.}
            \item {With careful data engineering (various variance stabilizing transformations, and splines) and model selection, our logistic regression model was able to achieve similar performance as tree-based GBMs while being much more interpretable and stable over distributional shifts}
            \item {Deployed the model on our internal scoring platform as a Python package.}
            \item {Guided by Agile principles: fast iterations of minimum viable product, quick adaptation to changes, and integration with business team for smooth communication.}\\
          \end{cvitems}
          \item {Independently developed a Python package to perform large-scale feature engineering in parallel with Dask. The package employs a graphical representation of feature lineages to allow easy parallelization and extraction of feature lineages. The package is now used across multiple teams to standardize feature generation and documentation.}
          \item {Proactively helped operations team to build a Monte Carlo simulation tool in Python to simulate customer call queue. It significantly improved call centre staffing to shorten customer wait time and reduce operations cost.}
          \item {Promoted to Senior Data Scientist in July 2017.}
        \end{cvitems}
      }

\end{cventries}
